\documentclass[10pt,a4paper]{moderncv}
\usepackage{ctex}
\usepackage{ragged2e}
\usepackage[scale=0.8]{geometry}
\moderncvstyle{banking}
\moderncvcolor{blue}
\usepackage{fontawesome}
\usepackage{lmodern,etaremune,bibentry,import,textcomp,fancyhdr,lastpage,multibib}
\usepackage[absolute,overlay]{textpos}
\usepackage[square,sort,comma,numbers]{natbib}
\usepackage[normalem]{ulem}

\usepackage{fontspec}                  %引入字体设置宏包
\setsansfont{Times New Roman}          %英文无衬线字体
\setmonofont{Times New Roman}          %英文等宽字体

\setcounter{page}{1}
\pagenumbering{arabic}
\pagestyle{fancy}
\fancyhf{}

% -----------------------------------------personal data-------------------------------------
\name{陈}{妮}
\title{简~历}

\phone[mobile]{+966~0544~018462}
% \phone[fixed]{+82~02~880~1699}
% \phone[mobile]{+86~137~6121~5137}
% \phone[fixed]{+86~021~6991~8371}
\email{ni.chen@kaust.edu.sa}
\homepage{http://ni-chen.github.io}

\extrainfo{更新于 \today}

\lfoot{\mdseries\slshape{\color{gray}{简历}}}
\rfoot{\mdseries\slshape{\color{gray}{陈妮}}}
\renewcommand{\footrulewidth}{0.1pt}{\color{gray}}
\cfoot{\color{color2}\itshape \thepage~/~\pageref{LastPage}}

%----------------------------content-----------------------------------------
\begin{document}
\hskip -2.2cm 
{\makecvtitle}
\begin{textblock}{0}(12.3,1.1)
	\includegraphics[width=2.2cm]{bio.jpg}
\end{textblock}

%----------------------------------------------------------------------------------------------
\vspace{-15pt}
\section{工作经历}
\cventry{2019年11月~$\sim$~~~~~~~~~~~~~~~~~~~~}{研究员}{阿卜杜拉国王科技大学}{沙特阿拉伯, 图瓦}{}{合作导师:Wolfgang Heidrich, IEEE Fellow}
\cventry{2017年09月~$\sim$~2019年10月}{BK21 助理教授}{首尔国立大学}{韩国, 首尔}{}{合作导师:Byoungho Lee, IEEE/OSA/SPIE/SID Fellow}
\cventry{2016年07月~$\sim$~2017年09月}{助理研究员, 副研究员}{中国科学院 上海光学精密机械研究所}{中国, 上海}{}{}
\cventry{2014年09月~$\sim$~2016年05月}{研究科学家}{香港大学}{香港}{}{合作导师:Edmund Y. Lam, IEEE/OSA/SPIE Fellow}
% \cventry{2007年8月~$\sim$~2008年5月}{上海启明软件有限公司, 上海, 中国}{实习生}{}{}{}
%----------------------------------------------------------------------------------------------
\section{教育经历}
\cventry{2010年09月~$\sim$~2014年08月}{电气与计算机工程}{工学博士~~~首尔国立大学}{韩国, 首尔}{\textit{GPA: 91.3/100}}{论文: Full Complex Wave Generation Methods Using Multiple Intensity Images.}
{导师: \href{http://oeqelab.snu.ac.kr/PROF}{Byoungho Lee} (韩国科学院院士; IEEE、OSA、SPIE、SID 会士)}
\vspace{3pt}

\cventry{2008年09月~$\sim$~2010年08月}{计算机与通信工程}{工学硕士~~~忠北国立大学}{韩国, 清州}{\textit{GPA: 98.44/100}}
{论文: Study on the 3D Hologram Synthesis based on the Integral Imaging and \\ the Resolution Enhancement. }
{导师: \href{http://osp.chungbuk.ac.kr/lab/pro.html}{Nam Kim} 和 \href{https://sites.google.com/site/3dlabinha/People/advisor}{Jae-Hyeung Park}}  
\vspace{3pt}

\cventry{2004年09月~$\sim$~2008年07月}{软件工程}{工学学士~~~哈尔滨工业大学(威海)}{中国, 威海}
{\textit{GPA: 88.83/100 (top 2).}}{}

%----------------------------------------------------------------------------------------------
\section{研究领域}
主要研究方向是计算成像, 包括:
\begin{itemize}
\item \textbf{三维/四维成像:} 数字全息, 非相干合成全息, 相位成像
\item \textbf{三维显示:} 光场成像和显示, 增强现实全息光学原件设计和制造
\end{itemize}
% \vspace{3pt}

%----------------------------------------------------------------------------------------------
\section{学术成果}
\subsection{引用统计}
\begin{itemize}
\item Web of Science ResearcherID:~{http://www.researcherid.com/rid/C-5537-2012}
	\begin{itemize}
	\item 总引用数: 585
	% \item 他引用数: 248
	\item h-index: 11
	\end{itemize}
\item Google Scholar:~{https://scholar.google.com/citations?user=adQED6IAAAAJ}
	\begin{itemize}
	\item 总引用数: 915
% 	\item h-index: 16
	\item i10-index: 16
	\end{itemize}
\end{itemize}
\nobibliography{mypub}
\bibliographystyle{plaincn}

\subsection{期刊论文}
\begin{etaremune}
    \item\bibentry{Wang2020}
    \item\bibentry{Chen2020}
    \item\bibentry{Chen2020TCL}
    
    \item[-] \textbf{\normalsize{第一作者和通讯作者(14)}}
	\item\bibentry{Xu2020Access}

	\item\bibentry{Chen2019OLE}
	\item\bibentry{Chen2019IRLA}
	
	\item\bibentry{Zhou2018OE} 
    \item\bibentry{Chen2018Sensors}
    \item\bibentry{Zhou2018JOPT} 
    \item\bibentry{HCWang2018AO}   
	
	\item\bibentry{CHWang2017AOS}
	\item\bibentry{Chen2017AO}  
	
	\item\bibentry{Chen2016PR}
    \item\bibentry{Chen2016AO}
    
    \item\bibentry{Chen2014JOSK} 
	
	\item\bibentry{Chen2010OE} 
    \item\bibentry{Chen2011OE} 

%--------------------------------------------------------------------
    \item[-] \textbf{\normalsize{共同作者(13)}} 
    \item\bibentry{Wu2020OE}
	\item\bibentry{BLi2020OE}
	\item\bibentry{RYang2020AppSci}
	\item\bibentry{YLi2020OE}
	
	\item\bibentry{Wan2019AO}
	\item\bibentry{Su2019ApplSci}	
	
	\item\bibentry{Lyu2017SR}	
	\item\bibentry{Ren2017OL}
	\item\bibentry{Ren2016AO}
	\item\bibentry{Li2014COL}
	\item\bibentry{Park2014JID}
	\item\bibentry{Hong2013OE}
	\item\bibentry{Hong2011AO}

	
\end{etaremune}

\subsection{会议论文(第一作者和通讯作者)}
\begin{etaremune}
    \item[-] \textbf{\normalsize{第一作者和通讯作者}}
    \item\bibentry{Chen2020DH}
    \item\bibentry{Chen2019DH}

    \item\bibentry{Chen2018IMID}
    \item\bibentry{Zhou2018DH}

    \item\bibentry{Zhou2017OIT}
    \item\bibentry{Chen2017DH} 

    \item\bibentry{Chen2015DH} 

    \item\bibentry{Chen2014DH} 

    \item\bibentry{Chen2013PC} 
    \item\bibentry{Chen2013FRINGE} 
    \item\bibentry{Chen2013OSKAM} 

    \item\bibentry{Chen2012PA} 
    \item\bibentry{Chen2012IMID} 
    \item\bibentry{Chen20123DSA} 

    \item\bibentry{Chen2011DHIP} 
    \item\bibentry{Chen2011IMID} 
    \item\bibentry{Chen2011COOC} 
    \item\bibentry{Chen2011DH} 
    \item\bibentry{Chen2011OSKAM} 

    \item\bibentry{Chen2010IMID} 
    \item\bibentry{Chen2010SPIE} 

    \item\bibentry{Chen2009AOSK} 
    \item\bibentry{Chen2009workshop} 
    \item\bibentry{Chen2009COOC} 

    %----------------------------
    \item[-] \textbf{\normalsize{共同作者}} 
    \item\bibentry{Wang2018DH}
    \item\bibentry{Wan2018SPIE}
    
    \item\bibentry{Wang2017DH} 

    \item\bibentry{Ren2015IST} 
    \item\bibentry{Ren2015DH} 

    \item\bibentry{Lee2014IPC} 

    \item\bibentry{Li2013OPTIC} 

    \item\bibentry{Lee2012DHIP} 
    \item\bibentry{Lee2012HOLOMET} 

    \item\bibentry{Yeom2011COOC} 

    \item\bibentry{Park2010DH} 
    \item\bibentry{Park2010SPIE} 

    \item\bibentry{Park2009SPIE} 
\end{etaremune}

\subsection{专利}
\begin{etaremune}
    \item\bibentry{Xu2018}
\end{etaremune}

% \normalsize{}
%----------------------------------------------------------------------------------------------
\section{研究项目(部分)}

% \begin{etaremune}
%     \item\bibentry{Situ_2018_SIOM}
%     \item\bibentry{Peng_2018_SINO}
%     \item\bibentry{Chen_2018_NSFC}
%     \item\bibentry{Chen_2017_NRF}
%     \item\bibentry{Chen_2017_NSFS}
%     \item\bibentry{Han_2017_KJW}
%     \item\bibentry{Situ_2016_CAS}
%     \item\bibentry{Lam_2014_NSFC}
%     \item\bibentry{Lee_2013_Samsung}
%     \item\bibentry{Lee_2012_NRF}
%     \item\bibentry{Lee_2012_Samsung}
%     \item\bibentry{Lee_2011_Samsung}
% \end{etaremune}
\cventry{2018年01月01日~$\sim$~2020年12月31日}
{中国国家自然科学基金青年基金~(NSFC)}{非相干条件下真实三维物体全息图的合成研究}
{\textyen~240,000}
{61705241}
{\underline{第一负责人}}

\cventry{2017年08月01日~$\sim$~2018年07月31日}
{韩国研究财团~(NRF) \& 中华人民共和国科学技术委员会(MOST)}
{中韩青年科学家交流}
{W~30,000,000}{}
{\underline{第一负责人}}

\cventry{2017年05月01日~$\sim$~2020年04月30日}
{上海市自然科学基金~(NSFS)}
{基于照相技术的高分辨率非相干全息合成术的研究}
{\textyen~200,000}
{17ZR1433800}
{\underline{第一负责人}}

\cventry{2018年01月01日~$\sim$~2020年12月31日}
{成都市国际合作}{面向AR视觉增强的全息再现质量提升研究}
{\textyen~400,000}
{61705241}
{\underline{韩方负责人}}

\cventry{2019年01月01日~$\sim$~2020年12月31日}
{深圳市国际合作项目~(NSFC)}{多维度显微成像技术}
{\textyen~2000,000}
{61705241}
{\underline{韩方负责人}}


\cventry{2018年01月01日~$\sim$~2020年12月31日}
{中国科学院上海光学精密机械研究所}
{散射成像}
{\textyen~3,090,000}
{}
{主要参与人}

\cventry{2018年01月01日~$\sim$~2020年12月31日}
{中德科学中心}
{中德联合实验室}
{\textyen~1,600,000}
{}
{主要参与人}

\cventry{2017~$\sim$~2018}
{中华人民共和国xxxxx}
{xxxxxxx(保密)}
{\textyen~2,000,000/年}
{}
{主要参与人}

\cventry{2016年08月01日~$\sim$~2021年07月31日}
{中国科学院前沿科学重点研究项目}
{光在复杂介质中超深度成像的问题研究}
{\textyen~2,500,000}
{QYZDB-SSW-JSC002}
{主要参与人}

% \cventry{2014年09月~$\sim$~2016年05月}{国家自然科学基金国际地区项目}{基于时间反演的扫描全息术和其在荧光生物成像中的应用}{HK\$110.25}{N\_HKU714/13}{参与}
% \vspace{6pt}

\cventry{2013年09月01日~$\sim$~2020年04月30日}
{韩国未来科学技术部(MSIP/KETI), Giga KOREA项目}
{数字全息基础技术研究}
{}
{}
{参与}
%
%\cventry{2012年12月01日~$\sim$~2013年11月30日}{韩国国家研究基金(NRF)}{超视角条件下的三维显示中视觉调节和增强设备的研究}{}{2012R1A6A3A03038820}{参与}\vspace{6pt}
%
\cventry{2012年05月~$\sim$~2013年04月}
{三星集团}{基于编码孔径的多视角图像生成}
{{KW}~900,000,000}
{}
{参与}

\cventry{2012年07月~$\sim$~2013年06月}
{三星集团}
{可穿戴式/透过式头盔显示}
{{KW}~800,000,000}
{}
{参与}
%
%\cventry{2011年05月~$\sim$~2012年06月}{三星先进技术研究部}{高密度光线场的真实三维显示技术的可行性研究}{}{}{参与}\vspace{6pt}

% \vspace{6pt}
% \cventry{2008年04月~$\sim$~2011年03月}{韩国教育科学技术部}{医学三维显示和图像处理的开发}{}{}{参与}

% \vspace{6pt}
% \cventry{2008年01月~$\sim$~2011年12月}{韩国教育科学技术部}{下一代全场景三维显示技术的研究和开发}{}{}{参与}

%------------------------------------Teaching------------------------------------

\section{教学和指导}
\cventry{2017年6月}
{3D Display: an overview}
{同济大学}
{}
{}
{}

\cventry{2016年7月~$\sim$2018年6月}
{协助指导: 王海超(博士生);周奥(硕士生);王彩虹(硕士生)}
{中国科学院上海光学精密机械研究所}
{}
{}
{}

\cventry{2014年9月~$\sim$~2016年5月}
{协助指导 : 任振波(博士生)}
{香港大学}
{}
{}
{去向:西北工业大学,助理教授}


\cventry{2017年12月}
{毕业论文指导 : Gang Li (博士生)}
{首尔国立大学}
{}
{}
{去向:Facebook Reality Lab,Scientist}

\cventry{2018年12月}
{毕业论文指导 : Changwon Jang (博士生)}
{首尔国立大学}
{}
{}
{去向:Facebook Reality Lab,Scientist}

\cventry{2019年11月~$\sim$~}
{协助指导 : Congli Wang (博士生)}
{阿卜杜拉国王科技大学}
{}
{}
{去向:加州大学伯克利分校,博士后}

%----------------------------------------------------------------------------------------------
\section{奖励和荣誉}
\cventry{2017年09月}
{中国激光杂志社}
{优秀论文}
{}
{\textit{中国}}
{}

% \cventry{2010年09月~$\sim$~2014年08月}
% {首尔国立大学}{Brain Korea 21 奖学金~(7次)}
% {}
% {\textit{韩国}}
% {}

\cventry{2012年12月}
{光学工程和量子电子光学实验室, 国家等离子应用系统创造新研究中心, 首尔国立大学}
{特别奖}
{}
{\textit{韩国}}
{}
\cventry{2012年08月}
{12th International Meeting on Information Display}{杰出海报论文奖}
{}
{\textit{韩国}}
{}
\cventry{2011年05月}
{18th Conference on Optoelectronics and Optical Communication}
{杰出论文奖}
{}
{\textit{韩国}}
{}
\cventry{2010~$\sim$~2012}
{首尔国立大学}
{卓越学术表现奖学金~(3次)}
{}
{\textit{韩国}}
{}

\cventry{2010年02月}
{忠北国立大学}
{Brain Korea 21 杰出硕士课程学生}
{}
{\textit{韩国}}
{}

% \cventry{2008年09月~$\sim$~2010年08月}
% {忠北国立大学}
% {Brain Korea 21 奖学金~(4次)}
% {}
% {\textit{韩国}}
% {}

\cventry{2008年07月}
{哈尔滨工业大学}
{优秀毕业生}
{}
{\textit{中国}}
{}

\cventry{2007年09月}
{哈尔滨工业大学}
{国家励志奖学金}
{}
{\textit{中国}}
{}

% \cventry{2006年11月}
% {山东}
% {山东报刊社优秀生}
% {}
% {\textit{中国}}
% {}

% \cventry{2006年09月}
% {软件学院, 哈尔滨工业大学}
% {优秀社会实践者}
% {}
% {\textit{中国}}
% {}

\cventry{2006年03月}
{哈尔滨工业大学}
{国家奖学金}
{}{\textit{中国}}
{}

% \cventry{2004~$\sim$~2008}
% {哈尔滨工业大学}
% {一等人民奖学金~(3次)}
% {}
% {\textit{中国}}
% {}

% \cventry{2004~$\sim$~2008}
% {哈尔滨工业大学}
% {二等人民奖学金~(2次)}
% {}
% {\textit{中国}}
% {}

% \cventry{2004~$\sim$~2008}
% {哈尔滨工业大学}
% {三等人民奖学金~(1次)}
% {}
% {\textit{中国}}
% {}

% \cventry{2004~$\sim$~2008}
% {哈尔滨工业大学}
% {三好学生~(3次)}
% {}
% {\textit{中国}}
% {}

%----------------------------------------------------------------------------------------------

\section{学术活动}

\cventry{Jun. 2017~$\sim$}
{《光学学报》}
{期刊主题编辑}
{}{\textit{}}{}

\textbf{会议程序委员会委员} \\
{\em 
美国光学学会主题会议, 数字全息和三维成像年会, \hfill 2020, 2021\\
}
\vspace{-9pt} 

\textbf{会议组织} \\
{\em 
International Conference on Optical Instrument and Technology~(光学仪器与技术国际会议), \hfill 2017\\ 
香山会议 - "计算光学成像科学基础研究:机遇和挑战", \hfill 2017\\
}
\vspace{-6pt}

\cventry{2018~$\sim$2019}
{OSA Technical Group Leadership, Image Sensing and Pattern Recognition~(IR)}
{委员}
{}{\textit{}}{}
	
	
\textbf{审稿人} \\
{\em Optics Letters, Optics Express, Applied Optics, Journal of Optical Society of America A, \\
Optics and Lasers in Engineering, Optics Communications, IEEE Access, Applied Science, \\ 
ETRI Journal, Optik, Journal of Information Display,	Chinese Journal of Lasers, Acta Optica Sinica \\
国家自然科学基金
} 
\hfill 2013~$\sim$


%----------------------------------------------------------------------------------------------

\section{专业和个人技能}
\begin{itemize}
\item \textbf{计算机技能:}
    \\编程语言: {\em C, C++, PHP, Python.}
    \\编程工具: {\em Matlab, Visual Studio, .NET.}
    % \\数据库: {\em Oracle, Microsoft Access, SQL server, My sql.}
    \\数学软件: {\em Wolfram Mathematica.}
    \\文字处理: {\em \LaTeX, Microsoft Office, Markdown.}
    \\其他: {\em Adobe Photoshop, Adobe Illustrator, Blender.}
\item \textbf{外语:}
    \\英文: {\em 专业熟练}
    \\韩文: {\em 日常熟练}
\end{itemize}

\section{参考人}
\begin{itemize}
\item \textbf{Byoungho Lee 博士} \\
	首尔国立大学, 电子系, 教授/系主任, 韩国科学院院士, OSA / IEEE / SPIE Fellow\\
	韩国,首尔,冠岳区,冠岳路1号(151-744) \\
	电话: (82) 2-880-7245 \\
	传真: (82) 2-873-9953 \\
	邮箱: byoungho@snu.ac.kr \\
	网址: {http://oeqelab.snu.ac.kr/PROF}
\end{itemize}
\vspace{3pt}

\begin{itemize}
\item \textbf{Edmund Y. Lam博士}\\
  香港大学电子和电气工程系, 教授,香港工程院院士, OSA / IEEE / SPIE Fellow\\
  香港,中西区,薄扶林道,香港大学,周亦卿楼 \\
  电话: (852) 2241-5942 \\
  传真: (852) 2559-8738 \\
  邮箱: elam@eee.hku.hk  \\
  网址: {http://www.eee.hku.hk/$\verb+~+$elam/}
\end{itemize}
\vspace{3pt}

% \begin{itemize}
% \item \textbf{司徒国海~博士} \\
%     中国科学院, 上海光学精密机械研究所, 研究员\\
%     中国,上海市,嘉定区 清河路390号 \\
%     电话: (86) 021-69918371 \\
%     邮箱: ghsitu@siom.ac.cn \\
%     网址: {http://www.escience.cn/people/situ/index.html;jsessionid=E9BFBC8B7AB1BE3FB0D57962EFAF329E-n2}
% \end{itemize}
% \vspace{6pt}

% \begin{itemize}
% \item \textbf{周常河~博士} \\
%     暨南大学, 光子技术研究院, 教授,OSA Fellow\\
%     中国,广州,黄埔大道西601号暨南大学曾宪梓科学馆 \\
%     电话: (86) 020-85222046 \\
%     邮箱: chazhou@mail.shcnc.ac.cn \\
%     网址: {https://ipt.jnu.edu.cn/jinan/article/show\_article.php?id=707}
% \end{itemize}

% \begin{itemize}
% \item \textbf{Nam Kim 博士}\\
% 	忠北国立大学 电子和计算机工程系 教授 \\
% 	韩国,忠清北道,清州市,西文区,忠大路1号(28644),E8-10栋  \\
% 	电话: (82) 43-261-2482 \\
% 	传真: (82) 43-274-6206  \\
% 	邮箱: namkim@chungbuk.ac.kr  \\
% 	网址: {http://osp.chungbuk.ac.kr/lab/pro.html}
% \end{itemize}

% \begin{itemize}
% \item \textbf{Jae-Hyeung Park 博士}\\
% 	仁荷大学 信息和通讯工程系 副教授 \\
% 	韩国,仁川市南区,仁荷路100号(402-751)\\
% 	电话: (82) 32-860-7432 \\
% 	邮箱: jh.park@inha.ac.kr \\
% 	网址: {https://sites.google.com/site/3dlabinha/People/advisor}
% \end{itemize}


% \begin{itemize}
% \item \textbf{Dr. Jae-Hyun Jung}\\
% 	Instructor in Ophthalmology, Harvard Medical School\\
% 	Investigator at Schepens Eye Research Institute/Mass. Eye \& Ear \\
% 	Harvard University \\
% 	Massachusetts eye and Ear 20 Staniford Street, Boston, MA 02114\\
% 	Phone: (1) 617-912-2525 \\
% 	Email: jaehyun\_jung@meei.harvard.edu\\
% 	Web: {http://scholar.harvard.edu/jaehyun\_jung}
% \end{itemize}
%-----------------------------------letter----------------------------------

%\section{简介}
%陈妮博士于2008年从哈尔滨工业大学(威海)取得软件工程学士学位,2010和2014从韩国忠北国立大学和首尔国立大学取得电子工程硕士和博士学位。2014年到2016年以研究科学家的身份在香港大学电子和机电工程系从事科学研究。2016年7月开始在中国科学院上海光学精密机械研究所工作,历任助理研究员,副研究员。2017年9月开始因中韩青年科学家交流计划在韩国首尔国立大学从事科研工作。她的研究以计算光学成像为主,包括光场、相位成像以及全息术。到目前为止发表学术论文四十余篇,承担多项国家和省部级科研项目。
\label{unknown}
\end{document}